\section{Conclusiones}
\PARstart Parte integral de este trabajo fue comprender la manera por la cual se estructura la red de Internet. 
En el transcurso de los experimentos logramos ver de manera más concreta cómo funciona una red real y sus diferencias
con las redes planteadas en la práctica, que son de una índole más teórica. 
En Internet, si bien teórica puede ocurrir que un paquete pase por distintos caminos para llegar a un destino, 
en la práctica no es tanto así debido a que en los experimentos realizados obtuvimos pocas variaciones en los resultados.

También es sorpresivo cómo no todos los routers no funcionan de la misma manera en cuanto a la forma de responder
a un mensaje de "no recibido" de un paquete ICMP. Inicialmente, habíamos esperado que en cierta forma todos nos respondieran
si es que no llegó el paquete, pero sin embargo, esto no ocurrió.

El camino geográfico trazado por el recorrido de los paquetes indican que hay 2 rutas comunes hacia Europa. Por un lado los paquetes 
con destino Europa Oriental y Asia central van directamente hacia un nodo en Italia para luego hacer los saltos correspondientes.
Por el otro lado para llegar a destinos que se encuentran en centro y norte de Europa (Alemania, UK, Islandia)
los paquetes primero pasan por Estados Unidos para luego hacer el salto hacia el continente Europeo.
Finalmente el trazado por el recorrido de los paquetes con destino a Australia tambien pasa por Estados Unidos.
