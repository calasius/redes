\section{Métodos y condiciones}
\PARstart La implementaci\'on de la herramienta que permite realizar un \verb+traceroute+ se desarrollo de la siguiete forma:

El calculo del $RTT$ se implemento tomando el tiempo antes de enviar el paquete $ICMP$, enviando un paquete $ICMP$, y cuando llega la respuesta $ICMP$ de tipo $time exceded$ tomando el tiempo. Luego al tiempo final se le resta el inicial y ese es el $RTT$ entre saltos. 

Debido a que el $RTT$ calculado de esta forma podria ser impresiso por diversos motivos, por ejemplo el paquete $ICMP$ de tipo $time exceded$ podria tardar mas tiempo del real porque en un momento dado hay mucha congestion en la red. Se decidio Enviar varios paquetes y realizar un promedio de los $RTT$ obtenidos por cada envio, obteniendo asi una muestra mas estable.

Para el c\'alculo de los outliers se normalizaron los datos y se compararon con los valores de la tabla $thompson$ $\tau$ modificada de acuerdo al $n$ de la muestra según la siguente fórmula.

\begin{equation}
  \tau = \cfrac{t_{\cfrac{\alpha}{2}} . (n - 1)}{\sqrt{n}\sqrt{n-2+t_{\cfrac{\alpha}{2}^2}}}
\end{equation}

\begin{itemize}
\item n es el numero de datos de la muestra
\item $t_{\cfrac{\alpha}{2}}$ es el valor crítico de t-student, donde $\alpha$ = 0.05 y df = n-2.
\end{itemize}
