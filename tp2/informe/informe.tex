\documentclass[%
        %draft,
        %submission,
        %compressed,
        final,
        %
        %technote,
        %internal,
        %submitted,
        %inpress,
        %reprint,
        %
        %titlepage,
        notitlepage,
        %anonymous,
        narroweqnarray,
        inline,
        %twoside,
        ]{ieee}

\usepackage{ieeefig}
\usepackage[utf8]{inputenc}
\usepackage{float}
\usepackage{booktabs}
\usepackage{caption}

\begin{document}

%----------------------------------------------------------------------
% Title Information, Abstract and Keywords
%----------------------------------------------------------------------
\title[Short Title]{Trabajo Práctico Nro 1: Wiretapping}

% format author this way for journal articles.
\author[SHORT NAMES]{%
      Christian Bonomi \small{LU:727/02},
    \and
      Claudio Gauna \small{LU:733/99},
    \and
      Marcelo Ferranti \small{LU:744/08},
    \and  
      Andrés Gutter \small{LU:007/08}
  }

% format author this way for conference proceeding

% specifiy the journal name
\journal{Teoría de las comunicaciones, 2016}

% Or, when the paper is a preprint, try this...
%\journal{IEEE Transactions on Something, 1997, TN\#9999.}

% Or, specify the conference place and date.

% make the title
\maketitle               

% do the abstract
\begin{abstract}

En este trabajo nos encargamos de implementar una serie de herramientas
para capturar los paquetes ARP de ciertas LANs, analizar las capturas
y tratar de inferir la topología de la red a partir de las mismas.
 
\end{abstract}

% do the keywords
\begin{keywords}
ARP, IP, MAC, Ethernet, Wi-Fi, Entropía, Fuente de información,
\end{keywords}

% start the main text ...
%----------------------------------------------------------------------
% SECTION I: Introduction
%----------------------------------------------------------------------

\section{Introducción}

\PARstart 
En el modelo OSI, la capa de enlace es la responsable de la transferencia fiable
de información a través de un circuito de transmisión de datos. Recibe peticiones de
una capa superior (la capa de red) y utiliza los servicios de la capa física (primer
capa). 
\begin{figure}[h]
  \centering
    \includegraphics[width=0.2\textwidth]{modeloOSI.png}
  \caption{}
  \label{}
\end{figure}
\\
ARP es un protocolo que se encarga de ser el nexo entre la capa de enlace y la capa de red. 
Su tarea es poder asociar direcciones IP (obtenidas por algún medio externo) con la dirección MAC (Media Access Control).
A nivel teórico, es una suerte de protocolo se encarga de encontrar la dirección MAC que corresponde a una IP determinada.
Para lograr dicho objetivo, un host emite un paquete en la red con dirección broadcast (ARP request) preguntando
por una IP específica esperando por la respuesta (ARP reply) de algún host conectado a esa red. 


Los paquetes ARP tienen el siguiente formato:
\begin{figure}[!h]
  \centering
    \includegraphics[width=0.45\textwidth]{arpPacket.png}
  \caption{}
  \label{}
\end{figure}
\\
Cuando un host A emite un paquete ARP request y luego el host B con dirección IP igual a la requerida por el host B
responde con el paquete ARP reply, el host A se guarda en una tabla de cache la dirección MAC del host B.
Sin embargo la tabla se elimina cada cierto tiempo ya que las direcciones pueden cambiar.

Nuestro objetivo será construir herramientas que nos permitan analizar este tráfico,
mas específicamente los mensajes ARP request de los hosts conectados a una red no trivial.

\section{Desarrollo}

\PARstart Para analizar la red se van a utilizar dos herramientas. Una que modela los paquetes Ethernet
capturados como una fuente de información binaria de memoria nula $S$, 
definiendo el conjunto de símbolos que emite como ${S_{BROADCAST}, S_{UNICAST}}$.
La segunda herramienta modela una fuente de información de memoria nula $S1$ 
que emite símbolos que son la IP destino de los paquetes $ARP$ de tipo $Who-has$ capturados.
Un símbolo de esta fuente será distinguido cuando la información provista
por el símbolo sea menor que la entropía de la fuente.

\subsection{Descripción y uso de la implementación}

TODO redactar una descripción y forma de uso de los script.
para citar se puede usar esto: \cite{wireshark}
Para escribir comandos usar esto: \textit{python entropia.py (file.pcap)}.

\section{Resultados}
\subsection{Servidor unak.is}

\begin{center}
\captionof{table}{rtt promedio entre saltos para unak.is}
\begin{tabular}{llllr}
\toprule
        &    &               &    &       rtt \\
host & ttl & ip & cc &           \\
\midrule
unak.is & 1  & empty & empty &  0.000000 \\
        & 2  & empty & empty &  0.000000 \\
        & 3  & empty & empty &  0.000000 \\
        & 4  & empty & empty &  0.000000 \\
        & 5  & 200.89.161.137 & AR &  0.074301 \\
        & 6  & 200.89.165.5 &  AR &  0.003707 \\
        & 7  & 200.89.165.250 & AR &  0.004276 \\
        & 8  & 190.216.88.33 & AR &  0.000000 \\
        &    & empty & empty &  0.000000 \\
        & 9  & 67.16.159.37 & US &  0.145690 \\
        & 10 & 64.57.20.73 & US &  0.005504 \\
        & 11 & 64.57.20.34 & US &  0.005281 \\
        & 12 & 64.57.20.37 & US &  0.010810 \\
        & 13 & empty & empty &  0.000000 \\
        & 14 & 109.105.97.142 & SE &  0.080042 \\
        & 15 & 109.105.97.138 & SE &  0.011689 \\
        & 16 & 109.105.97.125 & SE &  0.004030 \\
        & 17 & 109.105.102.1 & SE &  0.049820 \\
        & 18 & 130.208.17.162 & IS &  0.009160 \\
        & 19 & 130.208.17.58 & IS &  0.001856 \\
        & 20 & 130.208.18.106 & IS &  0.019887 \\
        & 21 & 130.208.224.102 & IS &  0.003169 \\
\bottomrule
\end{tabular}
\end{center}

\begin{center}
\begin{tabular}{p{6.5cm}r}
Porcentaje de saltos que no responden los $Time$ $exceeded$: & \textbf{28\%} \\ \\ 
Largo de la ruta en términos de saltos que responden: &\textbf{15 saltos} \\ \\
Cantidad de enlaces intercontinentales: & \textbf{1} \\ \\
Cantidad de outliers según el método de Cimbala: & \textbf{1} \\ \\
\end{tabular}
\end{center}

Los outliers se corresponden con los enlaces intercontinentales.


\begin{figure}[H]
  \centering
    \includegraphics[width=0.45\textwidth]{histogramas_rtt/unak-is.png}
  \caption{RTT entre saltos}
  \label{entropia-s}
\end{figure}

\begin{center}
\captionof{table}{Outliers para unak.is}
\begin{tabular}{llllr}
\toprule
        &    &               &    &      rtt \\
host & ttl & ip & cc &          \\
\midrule
unak.is & 9  & 67.16.159.37 & US &  3.44859 \\
\bottomrule
\end{tabular}
\end{center}

\begin{figure}[H]
  \centering
    \includegraphics[width=0.45\textwidth]{histogramas_thompson/unak-is.png}
  \caption{RTT }
  \label{entropia-s}
\end{figure}

\begin{figure}[H]
  \centering
    \includegraphics[width=0.45\textwidth]{grafico-rutas/unak-is.png}
  \caption{Gráfico de la ruta}
  \label{entropia-s}
\end{figure}



\subsection{Servidor unis.no}
\begin{center}
\captionof{table}{rtt promedios entre saltos para unis.no}
\begin{tabular}{llllr}
\toprule
        &    &                &    &       rtt \\
host & ttl & ip & cc &           \\
\midrule
unis.no & 1  & empty & empty &  0.000000 \\
        & 2  & empty & empty &  0.000000 \\
        & 3  & empty & empty &  0.000000 \\
        & 4  & empty & empty &  0.000000 \\
        & 5  & 200.89.161.97 & AR &  0.072603 \\
        & 6  & 200.89.165.1 & AR &  0.005187 \\
        & 7  & 200.89.165.250 & AR &  0.003868 \\
        & 8  & empty & empty &  0.000000 \\
        & 9  & 64.215.102.181 & US &  0.137360 \\
        & 10 & 64.57.20.73 & US &  0.008581 \\
        & 11 & 64.57.20.34 & US &  0.006402 \\
        & 12 & 64.57.20.37 & US &  0.008664 \\
        & 13 & empty & empty &  0.000000 \\
        & 14 & 109.105.97.142 & SE &  0.085971 \\
        & 15 & 109.105.97.138 & SE &  0.010124 \\
        & 16 & 109.105.102.66 & SE &  0.019075 \\
        & 17 & 109.105.102.67 & SE &  0.012499 \\
        & 18 & 128.39.255.121 & NO &  0.004408 \\
        & 19 & 128.39.255.47 & NO &  0.019374 \\
        & 20 & 128.39.255.211 & NO &  0.009346 \\
        & 21 & 128.39.255.19 & NO &  0.006287 \\
        & 22 & 128.39.254.85 & NO &  0.008778 \\
        & 23 & 128.39.47.158 & NO &  0.008459 \\
        & 24 & 158.39.149.250 & NO &  0.005062 \\
\bottomrule
\end{tabular}
\end{center}

\begin{center}
\begin{tabular}{p{6.5cm}r}
Porcentaje de saltos que no responden los $Time$ $exceeded$: & \textbf{25\%} \\ \\ 
Largo de la ruta en términos de saltos que responden: &\textbf{18 saltos} \\ \\
Cantidad de enlaces intercontinentales: & \textbf{1} \\ \\
Cantidad de outliers según el método de Cimbala: & \textbf{2} \\ \\
\end{tabular}
\end{center}

No se corresponden los enlaces con los outliers.

\begin{figure}[H]
  \centering
    \includegraphics[width=0.45\textwidth]{histogramas_rtt/unis-no.png}
  \caption{RTT entre saltos}
  \label{entropia-s}
\end{figure}

\begin{center}\captionof{table}{Outliers para unis.no}

\begin{tabular}{llllr}
\toprule
        &    &                &    &       rtt \\
host & ttl & ip & cc &           \\
\midrule
unis.no & 9  & 64.215.102.181 & US &  3.598609 \\
        & 14 & 109.105.97.142 & SE &  2.049258 \\
\bottomrule
\end{tabular}

\end{center}

\begin{figure}[H]
  \centering
    \includegraphics[width=0.45\textwidth]{histogramas_thompson/unis-no.png}
  \caption{RTTs Normailzados comparados con el valor Thompson}
  \label{entropia-s}
\end{figure}

\begin{figure}[H]
  \centering
    \includegraphics[width=0.45\textwidth]{grafico-rutas/unis-no.png}
  \caption{Gráfico de la ruta}
  \label{entropia-s}
\end{figure}




\subsection{Servidor www.fu-berlin.de}
\begin{center}
\captionof{table}{rtt promedios entre saltos para www.fu-berlin.de}
\begin{tabular}{llllr}
\toprule
                 &    &                &    &       rtt \\
host & ttl & ip & cc &           \\
\midrule
www.fu-berlin.de & 1  & empty & empty &  0.000000 \\
                 & 2  & empty & empty &  0.000000 \\
                 & 3  & empty & empty &  0.000000 \\
                 & 4  & empty & empty &  0.000000 \\
                 & 5  & 200.89.161.97 & AR &  0.076244 \\
                 & 6  & 200.89.165.197 & AR &  0.003247 \\
                 & 7  & 200.89.165.222 & AR &  0.005229 \\
                 & 8  & empty & empty &  0.000000 \\
                 & 9  & 67.17.94.249 & US &  0.131693 \\
                 & 10 & empty & empty &  0.000000 \\
                 & 11 & 4.69.154.73 & DE &  0.093868 \\
                 & 12 & 4.69.154.73 & DE &  0.011529 \\
                 & 13 & 212.162.4.6 & DE &  0.006355 \\
                 & 14 & 188.1.144.134 & DE &  0.014622 \\
                 & 15 & 188.1.234.174 & DE &  0.013845 \\
                 & 16 & 160.45.252.102 & DE &  0.009071 \\
                 & 17 & 130.133.99.102 & DE &  0.001297 \\
                 & 18 & 130.133.99.110 & DE&  0.004726 \\
                 & 19 & 160.45.170.10 & DE &  0.014093 \\
\bottomrule
\end{tabular}

\end{center}

\begin{center}
\begin{tabular}{p{6.5cm}r}
Porcentaje de saltos que no responden los $Time$ $exceeded$: & \textbf{31\%} \\ \\ 
Largo de la ruta en términos de saltos que responden: &\textbf{13 saltos} \\ \\
Cantidad de enlaces intercontinentales: & \textbf{1} \\ \\
Cantidad de outliers según el método de Cimbala: & \textbf{2} \\ \\
\end{tabular}
\end{center}

No se corresponden los enlaces con los outliers.

\begin{figure}[H]
  \centering
    \includegraphics[width=0.45\textwidth]{histogramas_rtt/www-fu-berlin-de.png}
  \caption{RTT entre saltos}
  \label{entropia-s}
\end{figure}

\begin{center}
\captionof{table}{Outliers para www.fu-berlin.de}
\begin{tabular}{llllr}
\toprule
                 &    &                &    &       rtt \\
host & ttl & ip & cc &           \\
\midrule
www.fu-berlin.de & 9  & 67.17.94.249 & US &  2.985575 \\
                 & 11 & 4.69.154.73 &  DE &  1.971715 \\
\bottomrule
\end{tabular}

\end{center}

\begin{figure}[H]
  \centering
    \includegraphics[width=0.45\textwidth]{histogramas_thompson/www-fu-berlin-de.png}
  \caption{RTTs Normailzados comparados con el valor Thompson}
  \label{entropia-s}
\end{figure}

\begin{figure}[H]
  \centering
    \includegraphics[width=0.45\textwidth]{grafico-rutas/www-fu-berlin-de.png}
  \caption{Gráfico de la ruta}
  \label{entropia-s}
\end{figure}




\subsection{Servidor www.kstu.kz}
\begin{center}
\captionof{table}{rtt promedios entre saltos para www.kstu.kz}
\begin{tabular}{llllr}
\toprule
            &    &               &    &       rtt \\
host & ttl & ip & cc &           \\
\midrule
www.kstu.kz & 1  & empty & empty &  0.000000 \\
            & 2  & empty & empty &  0.000000 \\
            & 3  & empty & empty &  0.000000 \\
            & 4  & empty & empty &  0.000000 \\
            & 5  & 200.89.161.145 & AR &  0.078752 \\
            & 6  & 200.89.165.150 & AR &  0.008279 \\
            & 7  & 185.70.203.20 & IT &  0.004113 \\
            & 8  & 195.22.214.131 & IT&  0.249391 \\
            & 9  & 195.22.211.109 & IT&  0.013956 \\
            & 10 & empty & empty &  0.000000 \\
            & 11 & 195.190.98.142 & RU &  0.081555 \\
            & 12 & 46.227.189.38 & KZ &  0.003893 \\
            &    & empty & empty &  0.000000 \\
            & 13 & 80.241.35.198 & KZ &  0.014117 \\
            & 14 & 85.29.137.2 & KZ &  0.068915 \\
            & 15 & 85.29.137.233 & KZ &  0.005942 \\
\bottomrule
\end{tabular}
\end{center}


\begin{center}
\begin{tabular}{p{6.5cm}r}
Porcentaje de saltos que no responden los $Time$ $exceeded$: & \textbf{40\%} \\ \\ 
Largo de la ruta en términos de saltos que responden: &\textbf{15 saltos} \\ \\
Cantidad de enlaces intercontinentales: & \textbf{1} \\ \\
Cantidad de outliers según el método de Cimbala: & \textbf{1} \\ \\
\end{tabular}
\end{center}


\begin{figure}[H]
  \centering
    \includegraphics[width=0.45\textwidth]{histogramas_rtt/www-kstu-kz.png}
  \caption{RTT entre saltos}
  \label{entropia-s}
\end{figure}

\begin{center}
\captionof{table}{Outliers para www.kstu.kz}
\begin{tabular}{llllr}
\toprule
            &    &               &    &       rtt \\
host & ttl & ip & cc &           \\
\midrule
www.kstu.kz & 8  & 195.22.214.131 & IT &  3.342247 \\
\bottomrule
\end{tabular}
\end{center}


\begin{figure}[H]
  \centering
    \includegraphics[width=0.45\textwidth]{histogramas_thompson/www-kstu-kz.png}
  \caption{RTTs Normailzados comparados con el valor Thompson}
  \label{entropia-s}
\end{figure}

\begin{figure}[H]
  \centering
    \includegraphics[width=0.45\textwidth]{grafico-rutas/www-kstu-kz.png}
  \caption{Gráfico de la ruta}
  \label{entropia-s}
\end{figure}




\subsection{Servidor www.uq.edu.au}
\begin{center}
\captionof{table}{rtt promedios entre saltos para www.uq.edu.au}
\begin{tabular}{llllr}
\toprule
              &    &                &    &       rtt \\
host & ttl & ip & cc &           \\
\midrule
www.uq.edu.au & 1  & empty & empty &  0.000000 \\
              & 2  & empty & empty &  0.000000 \\
              & 3  & empty & empty &  0.000000 \\
              & 4  & empty & empty &  0.000000 \\
              & 5  & 200.89.161.81 & AR &  0.069609 \\
              & 6  & 200.89.165.1 & AR &  0.005534 \\
              & 7  & 200.89.165.250 & AR &  0.007442 \\
              & 8  & empty & empty &  0.000000 \\
              & 9  & 67.17.94.249 & US &  0.121653 \\
              & 10 & empty & empty &  0.000000 \\
              & 11 & empty & empty &  0.000000 \\
              & 12 & 4.68.127.54 & US &  0.011813 \\
              & 13 & 129.250.4.250 & US &  0.008370 \\
              & 14 & 129.250.2.219 & US &  0.027036 \\
              & 15 & 129.250.7.69 & US &  0.015778 \\
              & 16 & 129.250.3.123 & US &  0.008138 \\
              & 17 & 204.1.253.166 & US &  0.028387 \\
              & 18 & 202.158.194.172 & AU &  0.139608 \\
              & 19 & 113.197.15.68 & AU &  0.002438 \\
              & 20 & 113.197.15.66 & AU &  0.006682 \\
              & 21 & 113.197.15.152 & AU &  0.023330 \\
              & 22 & 113.197.15.4 & AU &  0.011024 \\
              & 23 & 113.197.15.7 & AU &  0.015266 \\
              & 24 & 113.197.15.34 & AU &  0.003360 \\
              & 25 & 138.44.129.46 & AU &  0.009573 \\
              & 26 & 130.102.159.1 & AU &  0.021310 \\
              & 27 & 130.102.0.242 & AU &  0.007072 \\
              & 28 & 130.102.82.63 & AU &  0.005245 \\
              & 29 & 130.102.131.70 & AU &  0.004558 \\
\bottomrule
\end{tabular}
\end{center}


\begin{center}
\begin{tabular}{p{6.5cm}r}
Porcentaje de saltos que no responden los $Time$ $exceeded$: & \textbf{24\%} \\ \\ 
Largo de la ruta en términos de saltos que responden: &\textbf{22 saltos} \\ \\
Cantidad de enlaces intercontinentales: & \textbf{1} \\ \\
Cantidad de outliers según el método de Cimbala: & \textbf{2} \\ \\
\end{tabular}
\end{center}

\begin{figure}[H]
  \centering
    \includegraphics[width=0.45\textwidth]{histogramas_rtt/www-uq-edu-au.png}
  \caption{RTT entre saltos}
  \label{entropia-s}
\end{figure}

\begin{center}
\captionof{table}{Outliers para www.uq.edu.au}
\begin{tabular}{llllr}
\toprule
              &    &                &    &       rtt \\
host & ttl & ip & cc &           \\
\midrule
www.uq.edu.au & 9  & 67.17.94.249 & US &  3.018553 \\
              & 18 & 202.158.194.172 & AU &  3.546898 \\
\bottomrule
\end{tabular}

\end{center}

\begin{figure}[H]
  \centering
    \includegraphics[width=0.45\textwidth]{histogramas_thompson/www-uq-edu-au.png}
  \caption{RTTs Normailzados comparados con el valor Thompson}
  \label{entropia-s}
\end{figure}

\begin{figure}[H]
  \centering
    \includegraphics[width=0.45\textwidth]{grafico-rutas/www-uq-edu-au.png}
  \caption{Gráfico de la ruta}
  \label{entropia-s}
\end{figure}





\subsection{Servidor auckland.ac.nz}

\begin{center}
\captionof{table}{rtt promedios entre saltos para auckland.ac.nz}
\begin{tabular}{llllr}
\toprule
               &    &               &    &       rtt \\
host & ttl & ip & cc &           \\
\midrule
auckland.ac.nz & 1  & empty & empty &  0.000000 \\
               & 2  & empty & empty &  0.000000 \\
               & 3  & empty & empty &  0.000000 \\
               & 4  & empty & empty &  0.000000 \\
               & 5  & 200.89.161.77 & AR &  0.076270 \\
               & 6  & 200.89.165.197 & AR &  0.009660 \\
               & 7  & 200.89.165.222 & AR &  0.010555 \\
               & 8  & empty & empty &  0.000000 \\
               & 9  & 67.17.94.249 & US &  0.116990 \\
               & 10 & empty & empty &  0.000000 \\
               & 11 & empty & empty &  0.000000 \\
               & 12 & 4.68.127.54 & US &  0.008144 \\
               & 13 & 129.250.4.250 & US &  0.008747 \\
               & 14 & 129.250.2.219 & US &  0.026343 \\
               & 15 & 129.250.7.69 & US &  0.013037 \\
               & 16 & 129.250.3.123 & US &  0.005543 \\
               & 17 & 204.1.253.166 & US &  0.045089 \\
               & 18 & 202.158.194.172 & AU &  0.120172 \\
               & 19 & 182.255.119.139 & AU &  0.014907 \\
               & 20 & 210.7.39.251 & NZ &  0.014551 \\
               & 21 & 210.7.39.178 & NZ &  0.004857 \\
\bottomrule
\end{tabular}
\end{center}

\begin{center}
\begin{tabular}{p{6.5cm}r}
Porcentaje de saltos que no responden los $Time$ $exceeded$: & \textbf{\%} \\ \\ 
Largo de la ruta en términos de saltos que responden: &\textbf{ saltos} \\ \\
Cantidad de enlaces intercontinentales: & \textbf{} \\ \\
Cantidad de outliers según el método de Cimbala: & \textbf{} \\ \\
\end{tabular}
\end{center}

\begin{figure}[H]
  \centering
    \includegraphics[width=0.45\textwidth]{histogramas_rtt/auckland-ac-nz.png}
  \caption{RTT entre saltos}
  \label{entropia-s}
\end{figure}

\begin{center}
\captionof{table}{Outliers para auckland.ac.nz}
\begin{tabular}{llllr}
\toprule
               &    &               &    &       rtt \\
host & ttl & ip & cc &           \\
\midrule
auckland.ac.nz & 9  & 67.17.94.249 & US &  3.078891 \\
               & 18 & 202.158.194.172 & AU &  3.176235 \\
\bottomrule
\end{tabular}

\end{center}

\begin{figure}[H]
  \centering
    \includegraphics[width=0.45\textwidth]{histogramas_thompson/auckland-ac-nz.png}
  \caption{RTTs Normalizados comparados con el valor Thompson}
  \label{entropia-s}
\end{figure}

\begin{figure}[H]
  \centering
    \includegraphics[width=0.45\textwidth]{grafico-rutas/auckland-ac-nz.png}
  \caption{Gráfico de la ruta}
  \label{entropia-s}
\end{figure}




\subsection{Servidor invertisuniversity.ac.in}

\begin{center}
\captionof{table}{rtt promedios entre saltos para invertisuniversity.ac.in}
\begin{tabular}{llllr}
\toprule
                         &    &               &    &       rtt \\
host & ttl & ip & cc &           \\
\midrule
invertisuniversity.ac.in & 1  & empty & empty &  0.000000 \\
                         & 2  & empty & empty &  0.000000 \\
                         & 3  & empty & empty &  0.000000 \\
                         & 4  & empty & empty &  0.000000 \\
                         & 5  & 200.89.161.145 & AR &  0.091815 \\
                         & 6  & 200.89.165.150 & AR&  0.013239 \\
                         & 7  & 185.70.203.20 & IT &  0.016529 \\
                         & 8  & 195.22.209.63 & IT &  0.227124 \\
                         & 9  & 149.3.183.130 & IT &  0.018799 \\
                         & 10 & empty & empty &  0.000000 \\
                         & 11 & 182.79.217.34 & IN &  0.131296 \\
                         & 12 & 202.56.223.241 & IN &  0.022426 \\
                         &    & empty & empty &  0.000000 \\
                         & 13 & 182.79.254.197 & IN &  0.019971 \\
                         & 14 & 125.16.26.235 & IN &  0.007836 \\
                         & 15 & 103.233.124.0 & IN &  0.026050 \\
                         & 16 & 103.233.124.254 & IN &  0.027040 \\
                         & 17 & 103.233.126.5 & IN &  0.017655 \\
                         &    & empty & empty &  0.000000 \\
                         & 18 & 103.233.126.17 & IN &  0.015503 \\
                         & 19 & 103.233.127.47 & IN &  0.011093 \\
                         & 20 & 103.241.180.132 & IN &  0.016200 \\
                         & 21 & 103.241.146.22 & IN &  0.012582 \\
\bottomrule
\end{tabular}

\end{center}

\begin{center}
\begin{tabular}{p{6.5cm}r}
Porcentaje de saltos que no responden los $Time$ $exceeded$: & \textbf{\%} \\ \\ 
Largo de la ruta en términos de saltos que responden: &\textbf{ saltos} \\ \\
Cantidad de enlaces intercontinentales: & \textbf{} \\ \\
Cantidad de outliers según el método de Cimbala: & \textbf{} \\ \\
\end{tabular}
\end{center}

\begin{figure}[H]
  \centering
    \includegraphics[width=0.45\textwidth]{histogramas_rtt/invertisuniversity-ac-in.png}
  \caption{RTT entre saltos}
  \label{entropia-s}
\end{figure}

\begin{center}
\captionof{table}{Outliers para invertisuniversity.ac.in}
\begin{tabular}{llllr}
\toprule
                         &    &               &    &       rtt \\
host & ttl & ip & cc &           \\
\midrule
invertisuniversity.ac.in & 8  & 195.22.209.63 & IT &  3.734289 \\
                         & 11 & 182.79.217.34 & IN &  1.924863 \\
\bottomrule
\end{tabular}

\end{center}

\begin{figure}[H]
  \centering
    \includegraphics[width=0.45\textwidth]{histogramas_thompson/invertisuniversity-ac-in.png}
  \caption{RTTs Normalizados comparados con el valor Thompson}
  \label{entropia-s}
\end{figure}

\begin{figure}[H]
  \centering
    \includegraphics[width=0.45\textwidth]{grafico-rutas/invertisuniversity-ac-in.png}
  \caption{Gráfico de la ruta}
  \label{entropia-s}
\end{figure}




\subsection{Servidor www.uae.ma}

\begin{center}
\captionof{table}{rtt promedios entre saltos para www.uae.ma}
\begin{tabular}{llllr}
\toprule
           &    &              &    &       rtt \\
host & ttl & ip & cc &           \\
\midrule
www.uae.ma & 1  & empty & empty &  0.000000 \\
           & 2  & empty & empty &  0.000000 \\
           & 3  & empty & empty &  0.000000 \\
           & 4  & empty & empty &  0.000000 \\
           & 5  & 200.89.161.129 & AR &  0.074943 \\
           & 6  & 200.89.165.5 & AR &  0.004708 \\
           & 7  & 200.89.165.250 & AR &  0.006304 \\
           & 8  & 195.22.220.102 & IT &  0.005447 \\
           & 9  & 195.22.219.3 & IT &  0.035226 \\
           & 10 &              & IT &  0.010738 \\
           & 11 & 149.3.181.65 & IT &  0.005863 \\
           & 12 & 129.250.2.227 & US &  0.113837 \\
           & 13 & 129.250.2.19 & US &  0.077303 \\
           & 14 & 129.250.4.141 & US &  0.008594 \\
           & 15 & 129.250.6.8 & US &  0.001446 \\
           & 16 & 81.25.207.146 & GB &  0.006293 \\
           & 17 & 50.97.18.211 & US &  0.009113 \\
           & 18 & 50.97.18.249 & US &  0.009749 \\
           & 19 & 159.253.158.131 & NL &  0.004349 \\
           &    & empty & empty &  0.000000 \\
           & 20 & 159.253.148.195 & NL &  0.006204 \\
\bottomrule
\end{tabular}

\end{center}

\begin{center}
\begin{tabular}{p{6.5cm}r}
Porcentaje de saltos que no responden los $Time$ $exceeded$: & \textbf{25\%} \\ \\ 
Largo de la ruta en términos de saltos que responden: &\textbf{15 saltos} \\ \\
Cantidad de enlaces intercontinentales: & \textbf{4} \\ \\
Cantidad de outliers según el método de Cimbala: & \textbf{} \\ \\
\end{tabular}
\end{center}


\begin{figure}[H]
  \centering
    \includegraphics[width=0.45\textwidth]{histogramas_rtt/www-uae-ma.png}
  \caption{RTT entre saltos}
  \label{entropia-s}
\end{figure}

\begin{center}
\captionof{table}{Outliers para www.uae.ma}
\begin{tabular}{llllr}
\toprule
           &    &              &    &       rtt \\
host & ttl & ip & cc &           \\
\midrule
www.uae.ma & 12 & 129.250.2.227 & US &  3.065712 \\
           & 13 & 129.250.2.19 & US &  1.895807 \\
\bottomrule
\end{tabular}

\end{center}

\begin{figure}[H]
  \centering
    \includegraphics[width=0.45\textwidth]{histogramas_thompson/www-uae-ma.png}
  \caption{RTTs Normalizados comparados con el valor Thompson}
  \label{entropia-s}
\end{figure}

\begin{figure}[H]
  \centering
    \includegraphics[width=0.45\textwidth]{grafico-rutas/www-uae-ma.png}
  \caption{Gráfico de la ruta}
  \label{entropia-s}
\end{figure}




\subsection{Servidor bifrost.is}

\begin{center}
\captionof{table}{rtt promedios entre saltos para bifrost.is}
\begin{tabular}{llllr}
\toprule
           &    &               &    &       rtt \\
host & ttl & ip & cc &           \\
\midrule
bifrost.is & 1  & empty & empty &  0.000000 \\
           & 2  & empty & empty &  0.000000 \\
           & 3  & empty & empty &  0.000000 \\
           & 4  & empty & empty &  0.000000 \\
           & 5  & 200.89.161.133 & AR &  0.077077 \\
           & 6  & 200.89.165.130 & AR &  0.004429 \\
           & 7  & 200.89.165.222 & AR &  0.006434 \\
           & 8  & 185.70.203.32 & IT &  0.003869 \\
           & 9  & 89.221.41.171 & IT &  0.128319 \\
           & 10 &               & IT &  0.006943 \\
           & 11 & 154.54.9.17 & US &  0.005663 \\
           & 12 & 154.54.24.233 & US &  0.003648 \\
           & 13 & 154.54.24.197 & US &  0.011277 \\
           & 14 & 154.54.24.221 & US &  0.009189 \\
           & 15 & 154.54.40.109 & US &  0.007895 \\
           & 16 & 154.54.42.86 & US &  0.091001 \\
           & 17 & 154.54.57.162 & US &  0.008998 \\
           & 19 & 149.6.98.110 & US &  0.013127 \\
           & 20 & 31.15.113.2 & IS &  0.013911 \\
           & 21 & 31.15.113.1 & IS &  0.008186 \\
           & 22 & 31.15.115.41 & IS &  0.007656 \\
           & 23 & 176.10.35.233 & IS &  0.031531 \\
           & 24 & 176.10.35.234 & IS &  0.004080 \\
           & 25 & 185.118.33.147 & IS &  0.007998 \\
\bottomrule
\end{tabular}

\end{center}

\begin{center}
\begin{tabular}{p{6.5cm}r}
Porcentaje de saltos que no responden los $Time$ $exceeded$: & \textbf{17\%} \\ \\ 
Largo de la ruta en términos de saltos que responden: &\textbf{20 saltos} \\ \\
Cantidad de enlaces intercontinentales: & \textbf{4} \\ \\
Cantidad de outliers según el método de Cimbala: & \textbf{} \\ \\
\end{tabular}
\end{center}

\begin{figure}[H]
  \centering
    \includegraphics[width=0.45\textwidth]{histogramas_rtt/bifrost-is.png}
  \caption{RTT entre saltos}
  \label{entropia-s}
\end{figure}

\begin{center}
\captionof{table}{Outliers para bifrost.is}
\begin{tabular}{llllr}
\toprule
           &    &               &    &       rtt \\
host & ttl & ip & cc &           \\
\midrule
bifrost.is & 9  & 89.221.41.171 & IT &  3.369697 \\
           & 16 & 154.54.42.86 & US &  2.221474 \\
\bottomrule
\end{tabular}

\end{center}

\begin{figure}[H]
  \centering
    \includegraphics[width=0.45\textwidth]{histogramas_thompson/bifrost-is.png}
  \caption{RTTs Normalizados comparados con el valor Thompson}
  \label{entropia-s}
\end{figure}

\begin{figure}[H]
  \centering
    \includegraphics[width=0.45\textwidth]{grafico-rutas/bifrost-is.png}
  \caption{Gráfico de la ruta}
  \label{entropia-s}
\end{figure}




\subsection{Servidor birmingham.ac.uk}

\begin{center}
\captionof{table}{rtt promedios entre saltos para birmingham.ac.uk}
\begin{tabular}{llllr}
\toprule
                 &    &             &    &       rtt \\
host & ttl & ip & cc &           \\
\midrule
birmingham.ac.uk & 1  & empty & empty &  0.000000 \\
                 & 2  & empty & empty &  0.000000 \\
                 & 3  & empty & empty &  0.000000 \\
                 & 4  & empty & empty &  0.000000 \\
                 & 5  & 200.89.161.85 & AR &  0.074961 \\
                 & 6  & 200.89.165.1 & AR &  0.004908 \\
                 & 7  & 200.89.165.250 & AR &  0.007433 \\
                 & 8  & empty & empty &  0.000000 \\
                 & 9  & 67.17.99.233 & US &  0.124818 \\
                 & 10 & empty & empty &  0.000000 \\
                 & 11 & empty & empty &  0.000000 \\
                 & 12 & 212.187.139.166 & GB &  0.090183 \\
                 & 13 & 146.97.33.2 & GB &  0.006402 \\
                 & 14 & 146.97.33.22 & GB &  0.006861 \\
                 & 15 & 146.97.37.154 & GB &  0.010930 \\
                 & 16 & 193.62.80.154 & GB &  0.005032 \\
                 & 17 & 193.62.80.174 & GB &  0.005705 \\
                 & 18 & 193.63.208.142 & GB &  0.007928 \\
                 & 19 & empty & empty &  0.000000 \\
\bottomrule
\end{tabular}

\end{center}

\begin{center}
\begin{tabular}{p{6.5cm}r}
Porcentaje de saltos que no responden los $Time$ $exceeded$: & \textbf{62\%} \\ \\ 
Largo de la ruta en términos de saltos que responden: &\textbf{11 saltos} \\ \\
Cantidad de enlaces intercontinentales: & \textbf{1} \\ \\
Cantidad de outliers según el método de Cimbala: & \textbf{3} \\ \\
\end{tabular}
\end{center}

\begin{figure}[H]
  \centering
    \includegraphics[width=0.45\textwidth]{histogramas_rtt/birmingham-ac-uk.png}
  \caption{RTT entre saltos}
  \label{entropia-s}
\end{figure}

\begin{center}
\captionof{table}{Outliers para birmingham.ac.uk}
\begin{tabular}{llllr}
\toprule
                 &    &             &    &       rtt \\
host & ttl & ip & cc &           \\
\midrule
birmingham.ac.uk & 5  & 200.89.161.85 & AR &  2.084216 \\
                 & 9  & 67.17.99.233 & US &  3.732069 \\
                 & 12 & 212.187.139.166 & GB &  2.587332 \\
\bottomrule
\end{tabular}

\end{center}

\begin{figure}[H]
  \centering
    \includegraphics[width=0.45\textwidth]{histogramas_thompson/birmingham-ac-uk.png}
  \caption{RTTs Normalizados comparados con el valor Thompson}
  \label{entropia-s}
\end{figure}

\begin{figure}[H]
  \centering
    \includegraphics[width=0.45\textwidth]{grafico-rutas/birmingham-ac-uk.png}
  \caption{Gráfico de la ruta}
  \label{entropia-s}
\end{figure}
\section{Conclusiones}
\PARstart Parte integral de este trabajo fue comprender la manera por la cual se estructura la red de Internet. 
En el transcurso de los experimentos logramos ver de manera más concreta cómo funciona una red real y sus diferencias
con las redes planteadas en la práctica, que son de una índole más teórica. 
En Internet, si bien teórica puede ocurrir que un paquete pase por distintos caminos para llegar a un destino, 
en la práctica no es tanto así debido a que en los experimentos realizados obtuvimos pocas variaciones en los resultados.

También es sorpresivo cómo no todos los routers no funcionan de la misma manera en cuanto a la forma de responder
a un mensaje de "no recibido" de un paquete ICMP. Inicialmente, habíamos esperado que en cierta forma todos nos respondieran
si es que no llegó el paquete, pero sin embargo, esto no ocurrió.

El camino geográfico trazado por el recorrido de los paquetes indican que hay 2 rutas comunes hacia Europa. Por un lado los paquetes 
con destino Europa Oriental y Asia central no pasan por Estados Unidos, sino que van directamente a un nodo en Italia 
para luego hacer los saltos correspondientes. Por el otro lado para llegar a destinos que se encuentran en centro y norte de Europa
(Alemania, UK, Islandia) los paquetes primero pasan por Estados Unidos para luego hacer el salto hacia el continente Europeo.
Finalmente el trazado por el recorrido de los paquetes con destino a Australia tambien pasa por Estados Unidos.

%\input{anexos.tex}

% do the biliography:

\bibliographystyle{IEEEbib}
\bibliography{my-bibliography-file}

\begin{thebibliography}{1}

\bibitem{wireshark}
http://www.wireshark.org/
\bibitem{scapy}
http://en.wikipedia.org/wiki/Scapy
\bibitem{python}
https://www.python.org/
\bibitem{ieee}
https://www.ieee.org
\end{thebibliography}


% where ``my-bibliography-file.bib'' is the name of the file with all the 
% BibTeX entries.

% do the biographies...
% If you want a picture with your biography, then specify the name of
% the postscript file in square brackets. That is, uncomment the
% following three lines and change the name of "face.ps" to the name of 
% your file.
%\begin{biography}[face.ps]{Gregory L. Plett}
%  A bio with a face...
%\end{biography}

%----------------------------------------------------------------------
% FIGURES
%----------------------------------------------------------------------
% There are many ways to include figures in the text. We will assume
% that the figure is some sort of EPS file.
%
% The outdated packages epsfig and psfig allow you to insert figures
% like: \psfig{filename.eps} These should really be done now using the
% \includegraphics{filename.eps} command.  
%
% i.e.,
%
% \includegraphics{file.eps}
%
% whenever you want to include the EPS file 'file.eps'. There are many
% options for the includegraphics command, and are outlined in the
% on-line documentation for the "graphics bundle". Using the options,
% you can specify the height, total height (height+depth), width, scale,
% angle, origin, bounding box "bb",view port, and can trim from around
% the sides of the figure. You can also force LaTeX to clip the EPS file
% to the bounding box in the file. I find that I often use the scale,
% trim and clip commands.
% 
% \includegraphics[scale=0.6,trim=0 0 0 0,clip=]{file.eps}
% 
% which magnifies the graphics by 0.6 (If I create a graphics for an
% overhead projector transparency, I find that a magnification of 0.6
% makes it look much better in a paper), trims 0 points off
% of the left, bottom, right and top, and clips the graphics. If the
% trim numbers are negative, space is added around the figure. This can
% be useful to help center the graphics, if the EPS file bounding box is
% not quite right.
% 
% To center the graphics,
% 
% \begin{center}
% \includegraphics...
% \end{center}
% 
% I have not yet written good documentation for this, but another 
% package which helps in figure management is the package ieeefig.sty,
% available at: http://www-isl.stanford.edu/people/glp/ieee.shtml
% Specify:
% 
%\usepackage{ieeefig} 
% 
% in the preamble, and whenever you want a figure,
% 
%\figdef{filename}
% 
% where, filename.tex is a LaTeX file which defines what the figure is.
% It may be as simple as
% 
% \inserteps{filename.eps}
%
% or
% \inserteps[includegraphics options]{filename.eps}
% 
% or may be a very complicated LaTeX file. 

\end{document}
