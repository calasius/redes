\section{Introducción}
\PARstart El objetivo principal de este trabajo práctico es averiguar de qué manera viaja un paquete que se envía 
desde un host hacia otro que está ubicado a una distancia geográfica considerable a través de Internet, 
que corresponde a la capa de Red dentro del modelo OSI.
Sabemos que cuando un host manda un paquete, éste último pasa a través de diferentes redes. Para lograr esto, 
vamos a crear una herramienta en Scapy que envía paquetes ICMP a una cierta dirección ip/web (de tres universidades ubicadas en el otro extremo del globo). 

Para dicho paquete, vamos a modificar incrementalmente el campo TTL (Time To Live). Éste es un campo que indica 
cuántos saltos entre red y red puede dar a lo sumo. Cuando no llega a destino debido a que no le alcanzó el TTL, 
retorna la cantidad de milisegundos que le tomó ir hasta donde pudo llegar y volver al host origen (RTT Round Trip Time)
y las coordenadas de geolocalización de la ip alcanzada.

\subsection{Instalación de librerias necesarias}

Para poder ejecutar la herramienta es necesario tener instaladas las siguientes librerias:

\subsubsection{GeoIP}
\begin{verbatim}
pip install pygeoip
\end{verbatim}

\subsubsection{Basemap}
\begin{verbatim}
sudo apt-get install python-mpltoolkits.basemap
\end{verbatim}